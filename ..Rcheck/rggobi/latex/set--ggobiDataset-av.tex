\HeaderA{"[<-.ggobiDataset"}{Assignments for ggobi datasets}{"[<.Rdash..ggobiDataset"}
\aliasA{[<\Rdash.ggobiDataset}{"[<-.ggobiDataset"}{[<.Rdash..ggobiDataset}
\keyword{manip}{"[<-.ggobiDataset"}
\keyword{internal}{"[<-.ggobiDataset"}
\begin{Description}\relax
\end{Description}
\begin{Usage}
\begin{verbatim}"[<-.ggobiDataset"(x, i, j, value)\end{verbatim}
\end{Usage}
\begin{Arguments}
\begin{ldescription}
\item[\code{x}] row indices
\item[\code{i}] column indices
\item[\code{j}] new values
\item[\code{value}] 
\end{ldescription}
\end{Arguments}
\begin{Details}\relax
This functions allow one to treat a ggobi dataset as if it were a local
data.frame.  One can extract and assign elements within the dataset.

This method works by retrieving the entire dataset into
R, subsetting that copy, and then returning any changes.

@argument ggobi dataset
@arguments row indices
@arguments column indices
@arguments new values
@keyword manip
@keyword internal
\end{Details}
\begin{Author}\relax
Hadley Wickham <h.wickham@gmail.com>
\end{Author}
\begin{Examples}
\begin{ExampleCode}g <- ggobi(mtcars)
x <- g["mtcars"]
x[1:5, 1:5]
x[1:5, 1] <- 1:5
x[1:5, 1:5]\end{ExampleCode}
\end{Examples}

