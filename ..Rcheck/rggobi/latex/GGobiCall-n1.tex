\HeaderA{.GGobiCall}{Wrappers for calling C routines in the R-ggobi library.}{.GGobiCall}
\aliasA{.GGobiC}{.GGobiCall}{.GGobiC}
\keyword{dynamic}{.GGobiCall}
\keyword{internal}{.GGobiCall}
\begin{Description}\relax
\end{Description}
\begin{Usage}
\begin{verbatim}.GGobiCall(.name, ..., .gobi = ggobi_get(), .test=TRUE)\end{verbatim}
\end{Usage}
\begin{Arguments}
\begin{ldescription}
\item[\code{.name}] the simple name of the C routine to be resolved
\item[\code{...}] the arguments that to be passed to the \code{\LinkA{.C}{.C}} or \code{\LinkA{.Call}{.Call}}
\item[\code{.gobi}] the ggobi instance identifier that is to be passed to the C routine as its last argument
\item[\code{.test}] 
\end{ldescription}
\end{Arguments}
\begin{Details}\relax
\code{.GGobiC} and \code{.GGobiCall} convert the name and then call
their C invocation counterparts.

These functions map the simple name of a C routine into the
package-specific version of that name.  These allow use to hide the
use a name \textit{mangling} scheme of our choosing for the C level
routines in the shared library/DLL that provides the glue between R
and ggobi.  This is useful for avoiding name conflicts with other C
code in R or other packages.  These are only of relevance to the
developers of this package and those working with its C code.

The mapping of the name to its corresponding C routine name
is done in conjunction with the pre-processor macro
\code{RS\_GGOBI}. These  must be synchronized.
\end{Details}
\begin{Value}
the same result as the corresponding \code{.C} and \code{.Call}
\end{Value}
\begin{Author}\relax
Hadley Wickham <h.wickham@gmail.com>
\end{Author}
\begin{SeeAlso}\relax
\code{\LinkA{.C}{.C}}, \code{\LinkA{.Call}{.Call}}
\end{SeeAlso}
\begin{Examples}
\begin{ExampleCode}\end{ExampleCode}
\end{Examples}

