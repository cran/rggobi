\HeaderA{dataset}{Get ggobi dataset.}{dataset}
\keyword{manip}{dataset}
\keyword{internal}{dataset}
\begin{Description}\relax
Get an object representing an internal ggobi dataset
\end{Description}
\begin{Usage}
\begin{verbatim}dataset(which, .gobi = ggobi_get())\end{verbatim}
\end{Usage}
\begin{Arguments}
\begin{ldescription}
\item[\code{which}] which dataset to retrieve, an integer for positional matching or a character to match by name
\item[\code{.gobi}] GGobi instance
\end{ldescription}
\end{Arguments}
\begin{Details}\relax
It is convenient to be able to refer to and operate on a ggobi
dataset as if it were a regular R dataset.  This function allows one to
get an \code{ggobiDataset} object that represents a particular
dataset.
\end{Details}
\begin{Value}
A list of \code{ggobiDataset} objects
\end{Value}
\begin{Author}\relax
Hadley Wickham <h.wickham@gmail.com>
\end{Author}
\begin{SeeAlso}\relax
\code{link\{.ggobi\}}
\end{SeeAlso}
\begin{Examples}
\begin{ExampleCode}\end{ExampleCode}
\end{Examples}

