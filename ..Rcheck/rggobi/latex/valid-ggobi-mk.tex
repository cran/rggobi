\HeaderA{valid\_ggobi}{Determines whether a reference to an internal ggobi object is valid}{valid.Rul.ggobi}
\keyword{dynamic}{valid\_ggobi}
\keyword{internal}{valid\_ggobi}
\begin{Description}\relax
\end{Description}
\begin{Usage}
\begin{verbatim}valid_ggobi(.gobi)\end{verbatim}
\end{Usage}
\begin{Arguments}
\begin{ldescription}
\item[\code{.gobi}] an object of class \code{ggobi} which refers to an internal ggobi instance.
\end{ldescription}
\end{Arguments}
\begin{Details}\relax
One can create multiple, independent ggobi instances within a single
R session and one can also remove them either programmatically or
via the GUI.  To be able to refer to these objects which are
actually C-level internal objects, one has a reference or handle
from an S object. Since the C level object can be destroyed while the S
object still refers to them, this function allows one to check whether the
internal object to which R refers is still in existence.

@arguments an object of class \code{ggobi} which refers to an internal ggobi instance.
@value \code{TRUE} if real object still exist, \code{FALSE} otherwise
@keyword dynamic
\end{Details}
\begin{Value}
\code{TRUE} if real object still exist, \code{FALSE} otherwise
\end{Value}
\begin{Author}\relax
Hadley Wickham <h.wickham@gmail.com>
\end{Author}
\begin{Examples}
\begin{ExampleCode}g <- ggobi(mtcars)
valid_ggobi(g)
close(g)
valid_ggobi(g) \end{ExampleCode}
\end{Examples}

