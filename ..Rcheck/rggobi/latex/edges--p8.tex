\HeaderA{"edges<-"}{Set edges}{"edges<.Rdash."}
\aliasA{edges<\Rdash}{"edges<-"}{edges<.Rdash.}
\keyword{manip}{"edges<-"}
\begin{Description}\relax
Set edges for a dataset.
\end{Description}
\begin{Usage}
\begin{verbatim}"edges<-"(x, value)\end{verbatim}
\end{Usage}
\begin{Arguments}
\begin{ldescription}
\item[\code{x}] ggobiDataset
\item[\code{value}] matrix or data frame of edges.  First column should be from edge, second column to edge.
\end{ldescription}
\end{Arguments}
\begin{Details}\relax
To remove edges, set edges to NULL.

@arguments ggobiDataset
@arguments matrix or data frame of edges.  First column should be from edge, second column to edge.
@keyword manip
\end{Details}
\begin{Author}\relax
Hadley Wickham <h.wickham@gmail.com>
\end{Author}
\begin{Examples}
\begin{ExampleCode}cc<-cor(t(swiss),use="p", method="s") 
ccd<-sqrt(2*(1-cc)) 
a <- which(lower.tri(ccd), arr.ind=TRUE)
src <- row.names(swiss)[a[,2]]
dest <- row.names(swiss)[a[,1]] 
weight <- as.vector(as.dist(ccd))
gg <- ggobi(swiss)
gg$cor <- data.frame(weight)
edges(gg$cor) <- cbind(src, dest)
edges(gg$cor)
edges(gg$cor) <- NULL\end{ExampleCode}
\end{Examples}

