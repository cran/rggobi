\HeaderA{ggobi}{New ggobi}{ggobi}
\keyword{dynamic}{ggobi}
\begin{Description}\relax
Creates a new ggobi instance
\end{Description}
\begin{Usage}
\begin{verbatim}ggobi(data, args=character(0), mode=character(0), name = deparse(sys.call()[[2]]), ...)\end{verbatim}
\end{Usage}
\begin{Arguments}
\begin{ldescription}
\item[\code{data}] the name of a file containing the data, or a data frame or matrix containing the values
\item[\code{args}] a character vector of command-line arguments
\item[\code{mode}] data format GGobi should expect to read the data from, if reading from a file.
\item[\code{name}] the name to use in GGobi for the dataset, if one is specified
\item[\code{...}] 
\end{ldescription}
\end{Arguments}
\begin{Details}\relax
Create a new instance of GGobi with or without new data.  Use
function whenever you want to create a new GGobi indepdent of the
others---they will not share linked plots.  If you want to add
another dataset to an existing ggobi, please see \code{\LinkA{[<\Rdash.ggobi}{[<.Rdash..ggobi}}

There are currently three basic types of functions that you
can use with rggobi:

\Itemize{
\item Data getting and setting: see \code{\LinkA{[.ggobi}{[.ggobi}}, and \code{\LinkA{[.ggobiDataset}{[.ggobiDataset}}
\item "Automatic" brushing: see \code{\LinkA{glyph\_colour}{glyph.Rul.colour}},
\code{\LinkA{glyph\_size}{glyph.Rul.size}},  \code{\LinkA{glyph\_type}{glyph.Rul.type}},
\code{\LinkA{shadowed}{shadowed}},    \code{\LinkA{excluded}{excluded}}, and the associated
setter functions.
\item Edge modifcation: see \code{\LinkA{edges}{edges}}, \code{\LinkA{edges<\Rdash}{edges<.Rdash.}},
\code{\LinkA{ggobi\_longitudinal}{ggobi.Rul.longitudinal}}
}

You will generally spend most of your time working with
\code{ggobDataset}s, you retrieve using \code{\LinkA{\$.ggobiDataset}{.Rdol..ggobiDataset}},
\code{\LinkA{[.ggobiDataset}{[.ggobiDataset}}, or \code{\LinkA{[[.ggobiDataset}{[[.ggobiDataset}}.
Most of the time these will operate like normal R datasets while
pointing to the data in GGobi so that all changes are kept in sync.
If you need to force a ggobiDaataset to be an R \code{data.frame} use
\code{\LinkA{as.data.frame}{as.data.frame}}.
\end{Details}
\begin{Value}
A ggobi object
\end{Value}
\begin{Author}\relax
Hadley Wickham <h.wickham@gmail.com>
\end{Author}
\begin{Examples}
\begin{ExampleCode}ggobi(ggobi.find.file("data", "flea.csv"))
ggobi(ggobi.find.file("data", "flea.xml"))
ggobi(mtcars)
mtcarsg <- ggobi_get()$mtcars
glyph_colour(mtcarsg)
glyph_colour(mtcarsg) <- ifelse(mtcarsg$cyl < 4, 1, 2)
glyph_size(mtcarsg) <- mtcarsg$cyl\end{ExampleCode}
\end{Examples}

