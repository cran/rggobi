\HeaderA{"[.ggobiDataset"}{Subsettting}{"[.ggobiDataset"}
\aliasA{\$.ggobiDataset}{"[.ggobiDataset"}{.Rdol..ggobiDataset}
\aliasA{[.ggobiDataset}{"[.ggobiDataset"}{[.ggobiDataset}
\aliasA{[[.ggobiDataset}{"[.ggobiDataset"}{[[.ggobiDataset}
\keyword{manip}{"[.ggobiDataset"}
\begin{Description}\relax
Subsetting for ggobi datasets
\end{Description}
\begin{Usage}
\begin{verbatim}"[.ggobiDataset"(x, i, j, drop=FALSE)\end{verbatim}
\end{Usage}
\begin{Arguments}
\begin{ldescription}
\item[\code{x}] arguments for generic data.frame subset function
\item[\code{i}] drop dimensions?
\item[\code{j}] 
\item[\code{drop}] 
\end{ldescription}
\end{Arguments}
\begin{Details}\relax
This functions allow one to treat a ggobi dataset as if it were a local
data.frame.  One can extract and assign elements within the dataset.

This method works by retrieving the entire dataset into
R, and then subsetting with R.
\end{Details}
\begin{Value}
desired subset from data.frame
\end{Value}
\begin{Author}\relax
Hadley Wickham <h.wickham@gmail.com>
\end{Author}
\begin{Examples}
\begin{ExampleCode}g <- ggobi(mtcars)
x <- g$mtcars
x[1:5, 1:5]
x[[1]]
x$cyl\end{ExampleCode}
\end{Examples}

