\HeaderA{"\$.ggobi"}{Get ggobi data.}{".Rdol..ggobi"}
\aliasA{\$.ggobi}{"\$.ggobi"}{.Rdol..ggobi}
\aliasA{[.ggobi}{"\$.ggobi"}{[.ggobi}
\keyword{manip}{"\$.ggobi"}
\begin{Description}\relax
Conveniently retrieve ggobi dataset.
\end{Description}
\begin{Usage}
\begin{verbatim}"$.ggobi"(x, i)\end{verbatim}
\end{Usage}
\begin{Arguments}
\begin{ldescription}
\item[\code{x}] GGobi object
\item[\code{i}] name of dataset to retrive
\end{ldescription}
\end{Arguments}
\begin{Details}\relax
It is convenient to be able to refer to and operate on a ggobi
dataset as if it were a regular R dataset.  This function allows one to
get an \code{ggobiDataset} object that represents a particular
dataset.
\end{Details}
\begin{Author}\relax
Hadley Wickham <h.wickham@gmail.com>
\end{Author}
\begin{Examples}
\begin{ExampleCode}g <- ggobi(ChickWeight)
g["cars"] <- mtcars
g[1:2]
g["ChickWeight"]
g["cars"]
g$cars\end{ExampleCode}
\end{Examples}

