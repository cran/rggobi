\documentclass{article}
\usepackage[ae,hyper]{Rd}
\begin{document}
\HeaderA{addVariable.ggobiDataset}{Add variable}{addVariable.ggobiDataset}
\aliasA{addVariable}{addVariable.ggobiDataset}{addVariable}
\keyword{manip}{addVariable.ggobiDataset}
\begin{Description}\relax
Add variable to a ggobiDataset
\end{Description}
\begin{Usage}
\begin{verbatim}addVariable.ggobiDataset(x, vals, name, ...)\end{verbatim}
\end{Usage}
\begin{Arguments}
\begin{ldescription}
\item[\code{x}] ggobiDataset
\item[\code{vals}] values to add
\item[\code{name}] name of column to add
\item[\code{...}] 
\end{ldescription}
\end{Arguments}
\begin{Details}\relax
\end{Details}
\begin{Author}\relax
Hadley Wickham <h.wickham@gmail.com>
\end{Author}
\begin{Examples}
\begin{ExampleCode}\end{ExampleCode}
\end{Examples}

\HeaderA{"as.data.frame.ggobiDataset"}{Conversion methods}{"as.data.frame.ggobiDataset"}
\aliasA{as.data.frame.ggobiDataset}{"as.data.frame.ggobiDataset"}{as.data.frame.ggobiDataset}
\aliasA{as.matrix.ggobiDataset}{"as.data.frame.ggobiDataset"}{as.matrix.ggobiDataset}
\keyword{manip}{"as.data.frame.ggobiDataset"}
\keyword{internal}{"as.data.frame.ggobiDataset"}
\begin{Description}\relax
Convert a ggobiDataset to a regular R data.frame or matrix
\end{Description}
\begin{Usage}
\begin{verbatim}"as.data.frame.ggobiDataset"(x, ...)\end{verbatim}
\end{Usage}
\begin{Arguments}
\begin{ldescription}
\item[\code{x}] ggobiDataset
\item[\code{...}] 
\end{ldescription}
\end{Arguments}
\begin{Details}\relax
\end{Details}
\begin{Author}\relax
Hadley Wickham <h.wickham@gmail.com>
\end{Author}
\begin{Examples}
\begin{ExampleCode}\end{ExampleCode}
\end{Examples}

\HeaderA{clean.ggobi}{Clean ggobi}{clean.ggobi}
\keyword{dynamic}{clean.ggobi}
\keyword{internal}{clean.ggobi}
\begin{Description}\relax
Clean arguments for ggobi
\end{Description}
\begin{Usage}
\begin{verbatim}clean.ggobi(x)\end{verbatim}
\end{Usage}
\begin{Arguments}
\begin{ldescription}
\item[\code{x}] vector
\end{ldescription}
\end{Arguments}
\begin{Details}\relax
Arguments for ggobi need to be in specific format.
This function helps ensure that.
\end{Details}
\begin{Author}\relax
Hadley Wickham <h.wickham@gmail.com>
\end{Author}
\begin{Examples}
\begin{ExampleCode}\end{ExampleCode}
\end{Examples}

\HeaderA{close.ggobi}{Close GGobi instance}{close.ggobi}
\keyword{dynamic}{close.ggobi}
\begin{Description}\relax
Terminates and discards a ggobi instance
\end{Description}
\begin{Usage}
\begin{verbatim}close.ggobi(con, ...)\end{verbatim}
\end{Usage}
\begin{Arguments}
\begin{ldescription}
\item[\code{con}] ggobi object to close
\item[\code{...}] ignored and for compatability generic function.
\end{ldescription}
\end{Arguments}
\begin{Details}\relax
This allows the caller to close a ggobi instance and discard the
resources it uses. The function closes the display windows and
variable panel window associated with this ggobi instance.
It also resets the default ggobi instance to be the last
one created.

@arguments ggobi object to close
@arguments ignored and for compatability generic function.
@keyword dynamic
\end{Details}
\begin{Author}\relax
Hadley Wickham <h.wickham@gmail.com>
\end{Author}
\begin{Examples}
\begin{ExampleCode}g1 <- ggobi(mtcars)
g2 <- ggobi(mtcars)
close(g2)
close(ggobi_get())\end{ExampleCode}
\end{Examples}

\HeaderA{"colorscheme<-"}{Set active colour scheme.}{"colorscheme<.Rdash."}
\aliasA{colorscheme<\Rdash}{"colorscheme<-"}{colorscheme<.Rdash.}
\keyword{color}{"colorscheme<-"}
\begin{Description}\relax
Specify the active color scheme in a GGobi instance or the  session options.
\end{Description}
\begin{Usage}
\begin{verbatim}"colorscheme<-"(x, value)\end{verbatim}
\end{Usage}
\begin{Arguments}
\begin{ldescription}
\item[\code{x}] GGobi object
\item[\code{value}] colour scheme to make active
\end{ldescription}
\end{Arguments}
\begin{Details}\relax
This makes a particular color scheme active within a GGobi instance.
\end{Details}
\begin{Value}
The name of the previously active color scheme.
\end{Value}
\begin{Author}\relax
Hadley Wickham <h.wickham@gmail.com>
\end{Author}
\begin{Examples}
\begin{ExampleCode}g <- ggobi(mtcars)
colorscheme(g) <- "Set1 8"
colorscheme(g) <- 1\end{ExampleCode}
\end{Examples}

\HeaderA{colorscheme}{Get active colour scheme}{colorscheme}
\keyword{color}{colorscheme}
\begin{Description}\relax
Get name of the active colour scheme
\end{Description}
\begin{Usage}
\begin{verbatim}colorscheme(x)\end{verbatim}
\end{Usage}
\begin{Arguments}
\begin{ldescription}
\item[\code{x}] GGobi object
\end{ldescription}
\end{Arguments}
\begin{Details}\relax
\end{Details}
\begin{Author}\relax
Hadley Wickham <h.wickham@gmail.com>
\end{Author}
\begin{Examples}
\begin{ExampleCode}g <- ggobi(mtcars)
colorscheme(g)\end{ExampleCode}
\end{Examples}

\HeaderA{connecting\_edges}{Get connecting edges}{connecting.Rul.edges}
\keyword{manip}{connecting\_edges}
\begin{Description}\relax
Get actual edges from application of edges dataset to target dataset.
\end{Description}
\begin{Usage}
\begin{verbatim}connecting_edges(x, y)\end{verbatim}
\end{Usage}
\begin{Arguments}
\begin{ldescription}
\item[\code{x}] target ggobi dataset
\item[\code{y}] ggobi dataset containing edges
\end{ldescription}
\end{Arguments}
\begin{Details}\relax
\end{Details}
\begin{Author}\relax
Hadley Wickham <h.wickham@gmail.com>
\end{Author}
\begin{Examples}
\begin{ExampleCode}\end{ExampleCode}
\end{Examples}

\HeaderA{create\_time\_edges\_for\_unit}{Create time edges for a unit}{create.Rul.time.Rul.edges.Rul.for.Rul.unit}
\keyword{dynamic}{create\_time\_edges\_for\_unit}
\keyword{internal}{create\_time\_edges\_for\_unit}
\begin{Description}\relax
Create time edges for a unit
\end{Description}
\begin{Usage}
\begin{verbatim}create_time_edges_for_unit(d, time)\end{verbatim}
\end{Usage}
\begin{Arguments}
\begin{ldescription}
\item[\code{d}] data frame of observations from one unit
\item[\code{time}] time variable
\end{ldescription}
\end{Arguments}
\begin{Details}\relax
\end{Details}
\begin{Author}\relax
Hadley Wickham <h.wickham@gmail.com>
\end{Author}
\begin{Examples}
\begin{ExampleCode}\end{ExampleCode}
\end{Examples}

\HeaderA{create\_time\_edges}{Create time edges}{create.Rul.time.Rul.edges}
\keyword{dynamic}{create\_time\_edges}
\keyword{internal}{create\_time\_edges}
\begin{Description}\relax
Generate the connecting edges for longitudinal data
\end{Description}
\begin{Usage}
\begin{verbatim}create_time_edges(data, time = "Time", obsUnit = "ID")\end{verbatim}
\end{Usage}
\begin{Arguments}
\begin{ldescription}
\item[\code{data}] the dataset from which to compute the edge connections.
\item[\code{time}] name/index of time variable
\item[\code{obsUnit}] name/index of ID variable
\end{ldescription}
\end{Arguments}
\begin{Details}\relax
This function takes a data frame containing records with repeated
measurements for a given observational unit and generates the
appropriate edge information so that consecutive time
points can be connected in a GGobi display. The output of this
function can be given to setEdges to enable
viewing these time series plots for individuals.

Currently this only works if the data are sorted by ID.
This will be fixed in the future as it requires re-ordering
the records from the original dataset.
\end{Details}
\begin{Value}
A matrix with 2 columns and as many rows as in original data.  Each row in this matrix gives a directed line segment connecting one point in time to the next point in time
\end{Value}
\begin{Author}\relax
Hadley Wickham <h.wickham@gmail.com>
\end{Author}
\begin{Examples}
\begin{ExampleCode}\end{ExampleCode}
\end{Examples}

\HeaderA{dataset}{Get ggobi dataset.}{dataset}
\keyword{manip}{dataset}
\keyword{internal}{dataset}
\begin{Description}\relax
Get an object representing an internal ggobi dataset
\end{Description}
\begin{Usage}
\begin{verbatim}dataset(which, .gobi = ggobi_get())\end{verbatim}
\end{Usage}
\begin{Arguments}
\begin{ldescription}
\item[\code{which}] which dataset to retrieve, an integer for positional matching or a character to match by name
\item[\code{.gobi}] GGobi instance
\end{ldescription}
\end{Arguments}
\begin{Details}\relax
It is convenient to be able to refer to and operate on a ggobi
dataset as if it were a regular R dataset.  This function allows one to
get an \code{ggobiDataset} object that represents a particular
dataset.
\end{Details}
\begin{Value}
A list of \code{ggobiDataset} objects
\end{Value}
\begin{Author}\relax
Hadley Wickham <h.wickham@gmail.com>
\end{Author}
\begin{SeeAlso}\relax
\code{link\{.ggobi\}}
\end{SeeAlso}
\begin{Examples}
\begin{ExampleCode}\end{ExampleCode}
\end{Examples}

\HeaderA{dim.ggobiDataset}{ggobiDataset dimensions}{dim.ggobiDataset}
\keyword{attribute}{dim.ggobiDataset}
\keyword{internal}{dim.ggobiDataset}
\begin{Description}\relax
Retrieve the dimension of an ggobi dataset
\end{Description}
\begin{Usage}
\begin{verbatim}dim.ggobiDataset(x)\end{verbatim}
\end{Usage}
\begin{Arguments}
\begin{ldescription}
\item[\code{x}] dataset
\end{ldescription}
\end{Arguments}
\begin{Details}\relax
\end{Details}
\begin{Author}\relax
Hadley Wickham <h.wickham@gmail.com>
\end{Author}
\begin{Examples}
\begin{ExampleCode}\end{ExampleCode}
\end{Examples}

\HeaderA{dimnames.ggobiDataset}{Get dimension names}{dimnames.ggobiDataset}
\keyword{attribute}{dimnames.ggobiDataset}
\keyword{internal}{dimnames.ggobiDataset}
\begin{Description}\relax
Get row and column names for a ggobiDataget
\end{Description}
\begin{Usage}
\begin{verbatim}dimnames.ggobiDataset(x)\end{verbatim}
\end{Usage}
\begin{Arguments}
\begin{ldescription}
\item[\code{x}] ggobiDataget
\end{ldescription}
\end{Arguments}
\begin{Details}\relax
\end{Details}
\begin{Author}\relax
Hadley Wickham <h.wickham@gmail.com>
\end{Author}
\begin{Examples}
\begin{ExampleCode}\end{ExampleCode}
\end{Examples}

\HeaderA{"edges<-"}{Set edges}{"edges<.Rdash."}
\aliasA{edges<\Rdash}{"edges<-"}{edges<.Rdash.}
\keyword{manip}{"edges<-"}
\begin{Description}\relax
Set edges for a dataset.
\end{Description}
\begin{Usage}
\begin{verbatim}"edges<-"(x, value)\end{verbatim}
\end{Usage}
\begin{Arguments}
\begin{ldescription}
\item[\code{x}] ggobiDataset
\item[\code{value}] matrix or data frame of edges.  First column should be from edge, second column to edge.
\end{ldescription}
\end{Arguments}
\begin{Details}\relax
To remove edges, set edges to NULL.

@arguments ggobiDataset
@arguments matrix or data frame of edges.  First column should be from edge, second column to edge.
@keyword manip
\end{Details}
\begin{Author}\relax
Hadley Wickham <h.wickham@gmail.com>
\end{Author}
\begin{Examples}
\begin{ExampleCode}cc<-cor(t(swiss),use="p", method="s") 
ccd<-sqrt(2*(1-cc)) 
a <- which(lower.tri(ccd), arr.ind=TRUE)
src <- row.names(swiss)[a[,2]]
dest <- row.names(swiss)[a[,1]] 
weight <- as.vector(as.dist(ccd))
gg <- ggobi(swiss)
gg$cor <- data.frame(weight)
edges(gg$cor) <- cbind(src, dest)
edges(gg$cor)
edges(gg$cor) <- NULL\end{ExampleCode}
\end{Examples}

\HeaderA{edges}{Get edges}{edges}
\keyword{manip}{edges}
\begin{Description}\relax
Get edges for a dataset
\end{Description}
\begin{Usage}
\begin{verbatim}edges(x)\end{verbatim}
\end{Usage}
\begin{Arguments}
\begin{ldescription}
\item[\code{x}] ggobi dataset
\end{ldescription}
\end{Arguments}
\begin{Details}\relax
\end{Details}
\begin{Value}
A matrix of edge pairs
\end{Value}
\begin{Author}\relax
Hadley Wickham <h.wickham@gmail.com>
\end{Author}
\begin{Examples}
\begin{ExampleCode}\end{ExampleCode}
\end{Examples}

\HeaderA{"excluded<-.ggobiDataset"}{Set excluded status}{"excluded<.Rdash..ggobiDataset"}
\aliasA{excluded<\Rdash}{"excluded<-.ggobiDataset"}{excluded<.Rdash.}
\methaliasA{excluded<\Rdash.ggobiDataset}{"excluded<-.ggobiDataset"}{excluded<.Rdash..ggobiDataset}
\keyword{dynamic}{"excluded<-.ggobiDataset"}
\begin{Description}\relax
Set the exclusion status of points.
\end{Description}
\begin{Usage}
\begin{verbatim}"excluded<-.ggobiDataset"(x, value)\end{verbatim}
\end{Usage}
\begin{Arguments}
\begin{ldescription}
\item[\code{x}] ggobiDataset
\item[\code{value}] logical vector
\end{ldescription}
\end{Arguments}
\begin{Details}\relax
If a point is excluded it is not drawn.
\end{Details}
\begin{Author}\relax
Hadley Wickham <h.wickham@gmail.com>
\end{Author}
\begin{SeeAlso}\relax
\code{\LinkA{excluded}{excluded}}
\end{SeeAlso}
\begin{Examples}
\begin{ExampleCode}\end{ExampleCode}
\end{Examples}

\HeaderA{excluded.ggobiDataset}{Get excluded status}{excluded.ggobiDataset}
\aliasA{excluded}{excluded.ggobiDataset}{excluded}
\keyword{dynamic}{excluded.ggobiDataset}
\begin{Description}\relax
Get the exclusion status of points.
\end{Description}
\begin{Usage}
\begin{verbatim}excluded.ggobiDataset(x)\end{verbatim}
\end{Usage}
\begin{Arguments}
\begin{ldescription}
\item[\code{x}] ggobiDataget
\end{ldescription}
\end{Arguments}
\begin{Details}\relax
If a point is excluded it is not drawn.
\end{Details}
\begin{Author}\relax
Hadley Wickham <h.wickham@gmail.com>
\end{Author}
\begin{SeeAlso}\relax
\code{\LinkA{excluded<\Rdash}{excluded<.Rdash.}}
\end{SeeAlso}
\begin{Examples}
\begin{ExampleCode}\end{ExampleCode}
\end{Examples}

\HeaderA{"[.ggobiDataset"}{Subsettting}{"[.ggobiDataset"}
\aliasA{\$.ggobiDataset}{"[.ggobiDataset"}{.Rdol..ggobiDataset}
\aliasA{[.ggobiDataset}{"[.ggobiDataset"}{[.ggobiDataset}
\aliasA{[[.ggobiDataset}{"[.ggobiDataset"}{[[.ggobiDataset}
\keyword{manip}{"[.ggobiDataset"}
\begin{Description}\relax
Subsetting for ggobi datasets
\end{Description}
\begin{Usage}
\begin{verbatim}"[.ggobiDataset"(x, i, j, drop=FALSE)\end{verbatim}
\end{Usage}
\begin{Arguments}
\begin{ldescription}
\item[\code{x}] arguments for generic data.frame subset function
\item[\code{i}] drop dimensions?
\item[\code{j}] 
\item[\code{drop}] 
\end{ldescription}
\end{Arguments}
\begin{Details}\relax
This functions allow one to treat a ggobi dataset as if it were a local
data.frame.  One can extract and assign elements within the dataset.

This method works by retrieving the entire dataset into
R, and then subsetting with R.
\end{Details}
\begin{Value}
desired subset from data.frame
\end{Value}
\begin{Author}\relax
Hadley Wickham <h.wickham@gmail.com>
\end{Author}
\begin{Examples}
\begin{ExampleCode}g <- ggobi(mtcars)
x <- g$mtcars
x[1:5, 1:5]
x[[1]]
x$cyl\end{ExampleCode}
\end{Examples}

\HeaderA{"\$.ggobi"}{Get ggobi data.}{".Rdol..ggobi"}
\aliasA{\$.ggobi}{"\$.ggobi"}{.Rdol..ggobi}
\aliasA{[.ggobi}{"\$.ggobi"}{[.ggobi}
\keyword{manip}{"\$.ggobi"}
\begin{Description}\relax
Conveniently retrieve ggobi dataset.
\end{Description}
\begin{Usage}
\begin{verbatim}"$.ggobi"(x, i)\end{verbatim}
\end{Usage}
\begin{Arguments}
\begin{ldescription}
\item[\code{x}] GGobi object
\item[\code{i}] name of dataset to retrive
\end{ldescription}
\end{Arguments}
\begin{Details}\relax
It is convenient to be able to refer to and operate on a ggobi
dataset as if it were a regular R dataset.  This function allows one to
get an \code{ggobiDataset} object that represents a particular
dataset.
\end{Details}
\begin{Author}\relax
Hadley Wickham <h.wickham@gmail.com>
\end{Author}
\begin{Examples}
\begin{ExampleCode}g <- ggobi(ChickWeight)
g["cars"] <- mtcars
g[1:2]
g["ChickWeight"]
g["cars"]
g$cars\end{ExampleCode}
\end{Examples}

\HeaderA{ggobi\_check\_structs}{Check structs}{ggobi.Rul.check.Rul.structs}
\keyword{programming}{ggobi\_check\_structs}
\keyword{internal}{ggobi\_check\_structs}
\begin{Description}\relax
Validates GGobi and Rggobi views of internal data structures
\end{Description}
\begin{Usage}
\begin{verbatim}ggobi_check_structs()\end{verbatim}
\end{Usage}
\begin{Arguments}
\end{Arguments}
\begin{Details}\relax
This function is called when the Rggobi library is loaded and it verifies
that the sizes of the different internal data structures for GGobi are the
same for both the GGobi shared library/DLL and the Rggobi package. This is
important as the two shared libraries are compiled separately and may have
different compilation flags, etc. that make them incompatible. This
function simply compares the sizes of the two views of the structures and
raises a warning if they do not agree.

Essentially, you should never notice this function. A warning implies that
you need to re-install Rggobi against the version of GGobi you are using.

@value TRUE if the sizes in the two libraries are the same, otherwise a named logical vector indicating which structures are different
\end{Details}
\begin{Value}
TRUE if the sizes in the two libraries are the same, otherwise a named logical vector indicating which structures are different
\end{Value}
\begin{Author}\relax
Hadley Wickham <h.wickham@gmail.com>
\end{Author}
\begin{Examples}
\begin{ExampleCode}\end{ExampleCode}
\end{Examples}

\HeaderA{ggobi\_count}{Get number of GGobis}{ggobi.Rul.count}
\keyword{dynamic}{ggobi\_count}
\begin{Description}\relax
Retrieves the number of ggobi instances within this session
\end{Description}
\begin{Usage}
\begin{verbatim}ggobi_count()\end{verbatim}
\end{Usage}
\begin{Arguments}
\end{Arguments}
\begin{Details}\relax
One or more ggobi instances can be created within an R session so that one
can simultaneously look at different datasets or have different views
of the same dataset.  This function returns the number of ggobis in existence.

The different ggobi instances are maintained as C  level structures.
This function accesses a variable that stores how many are in existence
when the function is invoked.
\end{Details}
\begin{Author}\relax
Hadley Wickham <h.wickham@gmail.com>
\end{Author}
\begin{Examples}
\begin{ExampleCode}ggobi_count()\end{ExampleCode}
\end{Examples}

\HeaderA{ggobi\_data\_set\_variables}{Set variable values}{ggobi.Rul.data.Rul.set.Rul.variables}
\keyword{manip}{ggobi\_data\_set\_variables}
\keyword{internal}{ggobi\_data\_set\_variables}
\begin{Description}\relax
Set the variable values for a column in a ggobiDataset
\end{Description}
\begin{Usage}
\begin{verbatim}ggobi_data_set_variables(x, values, var, update = TRUE)\end{verbatim}
\end{Usage}
\begin{Arguments}
\begin{ldescription}
\item[\code{x}] ggobiDataset
\item[\code{values}] values of new variable
\item[\code{var}] variable name
\item[\code{update}] update?
\end{ldescription}
\end{Arguments}
\begin{Details}\relax
\end{Details}
\begin{Author}\relax
Hadley Wickham <h.wickham@gmail.com>
\end{Author}
\begin{Examples}
\begin{ExampleCode}\end{ExampleCode}
\end{Examples}

\HeaderA{ggobi.find.file}{Find GGobi file.}{ggobi.find.file}
\keyword{dynamic}{ggobi.find.file}
\keyword{internal}{ggobi.find.file}
\begin{Description}\relax
Finds a file stored somewhere in the ggobi installation.
\end{Description}
\begin{Usage}
\begin{verbatim}ggobi.find.file(..., check = F)\end{verbatim}
\end{Usage}
\begin{Arguments}
\begin{ldescription}
\item[\code{...}] bits of the path to join together
\item[\code{check}] 
\end{ldescription}
\end{Arguments}
\begin{Details}\relax
\end{Details}
\begin{Author}\relax
Hadley Wickham <h.wickham@gmail.com>
\end{Author}
\begin{Examples}
\begin{ExampleCode}ggobi.find.file("data","tips.xml")\end{ExampleCode}
\end{Examples}

\HeaderA{ggobi}{New ggobi}{ggobi}
\keyword{dynamic}{ggobi}
\begin{Description}\relax
Creates a new ggobi instance
\end{Description}
\begin{Usage}
\begin{verbatim}ggobi(data, args=character(0), mode=character(0), name = deparse(sys.call()[[2]]), ...)\end{verbatim}
\end{Usage}
\begin{Arguments}
\begin{ldescription}
\item[\code{data}] the name of a file containing the data, or a data frame or matrix containing the values
\item[\code{args}] a character vector of command-line arguments
\item[\code{mode}] data format GGobi should expect to read the data from, if reading from a file.
\item[\code{name}] the name to use in GGobi for the dataset, if one is specified
\item[\code{...}] 
\end{ldescription}
\end{Arguments}
\begin{Details}\relax
Create a new instance of GGobi with or without new data.  Use
function whenever you want to create a new GGobi indepdent of the
others---they will not share linked plots.  If you want to add
another dataset to an existing ggobi, please see \code{\LinkA{[<\Rdash.ggobi}{[<.Rdash..ggobi}}

There are currently three basic types of functions that you
can use with rggobi:

\Itemize{
\item Data getting and setting: see \code{\LinkA{[.ggobi}{[.ggobi}}, and \code{\LinkA{[.ggobiDataset}{[.ggobiDataset}}
\item "Automatic" brushing: see \code{\LinkA{glyph\_colour}{glyph.Rul.colour}},
\code{\LinkA{glyph\_size}{glyph.Rul.size}},  \code{\LinkA{glyph\_type}{glyph.Rul.type}},
\code{\LinkA{shadowed}{shadowed}},    \code{\LinkA{excluded}{excluded}}, and the associated
setter functions.
\item Edge modifcation: see \code{\LinkA{edges}{edges}}, \code{\LinkA{edges<\Rdash}{edges<.Rdash.}},
\code{\LinkA{ggobi\_longitudinal}{ggobi.Rul.longitudinal}}
}

You will generally spend most of your time working with
\code{ggobDataset}s, you retrieve using \code{\LinkA{\$.ggobiDataset}{.Rdol..ggobiDataset}},
\code{\LinkA{[.ggobiDataset}{[.ggobiDataset}}, or \code{\LinkA{[[.ggobiDataset}{[[.ggobiDataset}}.
Most of the time these will operate like normal R datasets while
pointing to the data in GGobi so that all changes are kept in sync.
If you need to force a ggobiDaataset to be an R \code{data.frame} use
\code{\LinkA{as.data.frame}{as.data.frame}}.
\end{Details}
\begin{Value}
A ggobi object
\end{Value}
\begin{Author}\relax
Hadley Wickham <h.wickham@gmail.com>
\end{Author}
\begin{Examples}
\begin{ExampleCode}ggobi(ggobi.find.file("data", "flea.csv"))
ggobi(ggobi.find.file("data", "flea.xml"))
ggobi(mtcars)
mtcarsg <- ggobi_get()$mtcars
glyph_colour(mtcarsg)
glyph_colour(mtcarsg) <- ifelse(mtcarsg$cyl < 4, 1, 2)
glyph_size(mtcarsg) <- mtcarsg$cyl\end{ExampleCode}
\end{Examples}

\HeaderA{ggobi\_get}{Get GGobi}{ggobi.Rul.get}
\keyword{dynamic}{ggobi\_get}
\begin{Description}\relax
Returns a ggobi reference
\end{Description}
\begin{Usage}
\begin{verbatim}ggobi_get(id = ggobi_count(), drop=TRUE)\end{verbatim}
\end{Usage}
\begin{Arguments}
\begin{ldescription}
\item[\code{id}] numeric vector indicating which ggobi instances to retrieve.  Use default if none specified
\item[\code{drop}] 
\end{ldescription}
\end{Arguments}
\begin{Details}\relax
This allows one to get a list of all the ggobi instances currently
in existence in the R session.  Also, one can fetch particular instances.

@arguments numeric vector indicating which ggobi instances to retrieve.  Use default if none specified
@returns list of ggobi instances
@keyword dynamic
\end{Details}
\begin{Author}\relax
Hadley Wickham <h.wickham@gmail.com>
\end{Author}
\begin{Examples}
\begin{ExampleCode}ggobi(mtcars)
ggobi(Nile)
ggobi_get(1)
ggobi_get(1:2)\end{ExampleCode}
\end{Examples}

\HeaderA{ggobi\_longitudinal}{Create longitudinal dataset.}{ggobi.Rul.longitudinal}
\keyword{dynamic}{ggobi\_longitudinal}
\begin{Description}\relax
Instantiate new ggobi with a longitudinal data set.
\end{Description}
\begin{Usage}
\begin{verbatim}ggobi_longitudinal(data, time = "Time", obsUnit = "ID")\end{verbatim}
\end{Usage}
\begin{Arguments}
\begin{ldescription}
\item[\code{data}] data frame
\item[\code{time}] time variable
\item[\code{obsUnit}] id variable
\end{ldescription}
\end{Arguments}
\begin{Details}\relax
@arguments data frame
@arguments time variable
@arguments id variable
@keyword dynamic
\end{Details}
\begin{Author}\relax
Hadley Wickham <h.wickham@gmail.com>
\end{Author}
\begin{Examples}
\begin{ExampleCode}data(Oxboys, package="nlme")
ggobi_longitudinal(Oxboys, "Occasion", "Subject")\end{ExampleCode}
\end{Examples}

\HeaderA{ggobi\_set\_data\_file}{Set data file.}{ggobi.Rul.set.Rul.data.Rul.file}
\keyword{manip}{ggobi\_set\_data\_file}
\keyword{internal}{ggobi\_set\_data\_file}
\begin{Description}\relax
Open data file and add to ggobi datasets.
\end{Description}
\begin{Usage}
\begin{verbatim}ggobi_set_data_file(file, mode = "unknown", add = TRUE, .gobi = ggobi_get())\end{verbatim}
\end{Usage}
\begin{Arguments}
\begin{ldescription}
\item[\code{file}] path to file
\item[\code{mode}] mode of file
\item[\code{add}] add?
\item[\code{.gobi}] ggobi instance
\end{ldescription}
\end{Arguments}
\begin{Details}\relax
\end{Details}
\begin{Author}\relax
Hadley Wickham <h.wickham@gmail.com>
\end{Author}
\begin{Examples}
\begin{ExampleCode}\end{ExampleCode}
\end{Examples}

\HeaderA{ggobi\_set\_data\_frame}{Set data frame.}{ggobi.Rul.set.Rul.data.Rul.frame}
\keyword{manip}{ggobi\_set\_data\_frame}
\keyword{internal}{ggobi\_set\_data\_frame}
\begin{Description}\relax
Add data.frame to ggobi instance.
\end{Description}
\begin{Usage}
\begin{verbatim}ggobi_set_data_frame(data, name = deparse(sys.call()[[2]]), description = paste("R data frame", name), id = rownames(data), .gobi = ggobi_get())\end{verbatim}
\end{Usage}
\begin{Arguments}
\begin{ldescription}
\item[\code{data}] data frame to add
\item[\code{name}] data set name (appears on tabs in ggobi)
\item[\code{description}] description of data frame
\item[\code{id}] rownames
\item[\code{.gobi}] ggobi instance
\end{ldescription}
\end{Arguments}
\begin{Details}\relax
\end{Details}
\begin{Author}\relax
Hadley Wickham <h.wickham@gmail.com>
\end{Author}
\begin{Examples}
\begin{ExampleCode}\end{ExampleCode}
\end{Examples}

\HeaderA{.ggobi.symbol}{Ggobi symbol}{.ggobi.symbol}
\keyword{dynamic}{.ggobi.symbol}
\keyword{internal}{.ggobi.symbol}
\begin{Description}\relax
\end{Description}
\begin{Usage}
\begin{verbatim}.ggobi.symbol(name)\end{verbatim}
\end{Usage}
\begin{Arguments}
\begin{ldescription}
\item[\code{name}] 
\end{ldescription}
\end{Arguments}
\begin{Details}\relax
Used entirely within R to map the given name to the name of the
corresponding C routine.

A simple way of generating the prefix for a symbol
used in this package/library so that we can hide
it from other packages and avoid conflicts.
\end{Details}
\begin{Value}
the name of the C routine corresponding to its argument
\end{Value}
\begin{Author}\relax
Hadley Wickham <h.wickham@gmail.com>
\end{Author}
\begin{Examples}
\begin{ExampleCode}\end{ExampleCode}
\end{Examples}

\HeaderA{ggobi\_version}{Get version}{ggobi.Rul.version}
\keyword{dynamic}{ggobi\_version}
\begin{Description}\relax
GGobi version information
\end{Description}
\begin{Usage}
\begin{verbatim}ggobi_version()\end{verbatim}
\end{Usage}
\begin{Arguments}
\end{Arguments}
\begin{Details}\relax
Return an object that describes the version of the ggobi
library being used. This allows code to execute
conditionally on certain version numbers, etc.

@value the release date of the ggobi library
@value a vector of 3 integers containing the major, minor and patch-level numbers.
@value a string version of the major, minor and patch-level numbers,
@keyword dynamic
\end{Details}
\begin{Value}
\begin{ldescription}
\item[\code{the release date of the ggobi library}] 
\item[\code{a vector of 3 integers containing the major, minor and patch-level numbers.}] 
\item[\code{a string version of the major, minor and patch-level numbers,}] 
\end{ldescription}
\end{Value}
\begin{Author}\relax
Hadley Wickham <h.wickham@gmail.com>
\end{Author}
\begin{Examples}
\begin{ExampleCode}ggobi_version()\end{ExampleCode}
\end{Examples}

\HeaderA{.GGobiCall}{Wrappers for calling C routines in the R-ggobi library.}{.GGobiCall}
\aliasA{.GGobiC}{.GGobiCall}{.GGobiC}
\keyword{dynamic}{.GGobiCall}
\keyword{internal}{.GGobiCall}
\begin{Description}\relax
\end{Description}
\begin{Usage}
\begin{verbatim}.GGobiCall(.name, ..., .gobi = ggobi_get(), .test=TRUE)\end{verbatim}
\end{Usage}
\begin{Arguments}
\begin{ldescription}
\item[\code{.name}] the simple name of the C routine to be resolved
\item[\code{...}] the arguments that to be passed to the \code{\LinkA{.C}{.C}} or \code{\LinkA{.Call}{.Call}}
\item[\code{.gobi}] the ggobi instance identifier that is to be passed to the C routine as its last argument
\item[\code{.test}] 
\end{ldescription}
\end{Arguments}
\begin{Details}\relax
\code{.GGobiC} and \code{.GGobiCall} convert the name and then call
their C invocation counterparts.

These functions map the simple name of a C routine into the
package-specific version of that name.  These allow use to hide the
use a name \textit{mangling} scheme of our choosing for the C level
routines in the shared library/DLL that provides the glue between R
and ggobi.  This is useful for avoiding name conflicts with other C
code in R or other packages.  These are only of relevance to the
developers of this package and those working with its C code.

The mapping of the name to its corresponding C routine name
is done in conjunction with the pre-processor macro
\code{RS\_GGOBI}. These  must be synchronized.
\end{Details}
\begin{Value}
the same result as the corresponding \code{.C} and \code{.Call}
\end{Value}
\begin{Author}\relax
Hadley Wickham <h.wickham@gmail.com>
\end{Author}
\begin{SeeAlso}\relax
\code{\LinkA{.C}{.C}}, \code{\LinkA{.Call}{.Call}}
\end{SeeAlso}
\begin{Examples}
\begin{ExampleCode}\end{ExampleCode}
\end{Examples}

\HeaderA{"glyph\_colour<-.ggobiDataset"}{Set glyph colour}{"glyph.Rul.colour<.Rdash..ggobiDataset"}
\aliasA{glyph\_color<\Rdash}{"glyph\_colour<-.ggobiDataset"}{glyph.Rul.color<.Rdash.}
\aliasA{glyph\_colour<\Rdash}{"glyph\_colour<-.ggobiDataset"}{glyph.Rul.colour<.Rdash.}
\methaliasA{glyph\_colour<\Rdash.ggobiDataset}{"glyph\_colour<-.ggobiDataset"}{glyph.Rul.colour<.Rdash..ggobiDataset}
\keyword{dynamic}{"glyph\_colour<-.ggobiDataset"}
\begin{Description}\relax
Set glyph colour
\end{Description}
\begin{Usage}
\begin{verbatim}"glyph_colour<-.ggobiDataset"(x, value)\end{verbatim}
\end{Usage}
\begin{Arguments}
\begin{ldescription}
\item[\code{x}] ggobiDataset
\item[\code{value}] vector of new colours
\end{ldescription}
\end{Arguments}
\begin{Details}\relax
\end{Details}
\begin{Author}\relax
Hadley Wickham <h.wickham@gmail.com>
\end{Author}
\begin{SeeAlso}\relax
\code{\LinkA{glyph\_colour}{glyph.Rul.colour}}
\end{SeeAlso}
\begin{Examples}
\begin{ExampleCode}\end{ExampleCode}
\end{Examples}

\HeaderA{glyph\_colour.ggobiDataset}{Get glyph colour}{glyph.Rul.colour.ggobiDataset}
\aliasA{glyph\_color}{glyph\_colour.ggobiDataset}{glyph.Rul.color}
\aliasA{glyph\_colour}{glyph\_colour.ggobiDataset}{glyph.Rul.colour}
\keyword{dynamic}{glyph\_colour.ggobiDataset}
\begin{Description}\relax
Get glyph colour
\end{Description}
\begin{Usage}
\begin{verbatim}glyph_colour.ggobiDataset(x)\end{verbatim}
\end{Usage}
\begin{Arguments}
\begin{ldescription}
\item[\code{x}] ggobiDataset
\end{ldescription}
\end{Arguments}
\begin{Details}\relax
\end{Details}
\begin{Author}\relax
Hadley Wickham <h.wickham@gmail.com>
\end{Author}
\begin{SeeAlso}\relax
\code{\LinkA{glyph\_colour<\Rdash}{glyph.Rul.colour<.Rdash.}}
\end{SeeAlso}
\begin{Examples}
\begin{ExampleCode}\end{ExampleCode}
\end{Examples}

\HeaderA{"glyph\_size<-.ggobiDataset"}{Set glyph size}{"glyph.Rul.size<.Rdash..ggobiDataset"}
\aliasA{glyph\_size<\Rdash}{"glyph\_size<-.ggobiDataset"}{glyph.Rul.size<.Rdash.}
\methaliasA{glyph\_size<\Rdash.ggobiDataset}{"glyph\_size<-.ggobiDataset"}{glyph.Rul.size<.Rdash..ggobiDataset}
\keyword{dynamic}{"glyph\_size<-.ggobiDataset"}
\begin{Description}\relax
Set glyph size
\end{Description}
\begin{Usage}
\begin{verbatim}"glyph_size<-.ggobiDataset"(x, value)\end{verbatim}
\end{Usage}
\begin{Arguments}
\begin{ldescription}
\item[\code{x}] ggobiDataset
\item[\code{value}] vector of new sizes
\end{ldescription}
\end{Arguments}
\begin{Details}\relax
\end{Details}
\begin{Author}\relax
Hadley Wickham <h.wickham@gmail.com>
\end{Author}
\begin{SeeAlso}\relax
\code{\LinkA{glyph\_size}{glyph.Rul.size}}
\end{SeeAlso}
\begin{Examples}
\begin{ExampleCode}\end{ExampleCode}
\end{Examples}

\HeaderA{glyph\_size.ggobiDataset}{Get glyph size}{glyph.Rul.size.ggobiDataset}
\aliasA{glyph\_size}{glyph\_size.ggobiDataset}{glyph.Rul.size}
\keyword{dynamic}{glyph\_size.ggobiDataset}
\begin{Description}\relax
Get glyph size
\end{Description}
\begin{Usage}
\begin{verbatim}glyph_size.ggobiDataset(x, value)\end{verbatim}
\end{Usage}
\begin{Arguments}
\begin{ldescription}
\item[\code{x}] ggobiDataset
\item[\code{value}] 
\end{ldescription}
\end{Arguments}
\begin{Details}\relax
\end{Details}
\begin{Author}\relax
Hadley Wickham <h.wickham@gmail.com>
\end{Author}
\begin{SeeAlso}\relax
\code{\LinkA{glyph\_size<\Rdash}{glyph.Rul.size<.Rdash.}}
\end{SeeAlso}
\begin{Examples}
\begin{ExampleCode}\end{ExampleCode}
\end{Examples}

\HeaderA{"glyph\_type<-.ggobiDataset"}{Set glyph type}{"glyph.Rul.type<.Rdash..ggobiDataset"}
\aliasA{glyph\_type<\Rdash}{"glyph\_type<-.ggobiDataset"}{glyph.Rul.type<.Rdash.}
\methaliasA{glyph\_type<\Rdash.ggobiDataset}{"glyph\_type<-.ggobiDataset"}{glyph.Rul.type<.Rdash..ggobiDataset}
\keyword{dynamic}{"glyph\_type<-.ggobiDataset"}
\begin{Description}\relax
Set glyph type
\end{Description}
\begin{Usage}
\begin{verbatim}"glyph_type<-.ggobiDataset"(x, value)\end{verbatim}
\end{Usage}
\begin{Arguments}
\begin{ldescription}
\item[\code{x}] ggobiDataset
\item[\code{value}] vector of new types
\end{ldescription}
\end{Arguments}
\begin{Details}\relax
\end{Details}
\begin{Author}\relax
Hadley Wickham <h.wickham@gmail.com>
\end{Author}
\begin{SeeAlso}\relax
\code{\LinkA{glyph\_type}{glyph.Rul.type}}
\end{SeeAlso}
\begin{Examples}
\begin{ExampleCode}\end{ExampleCode}
\end{Examples}

\HeaderA{glyph\_type.ggobiDataset}{Get glyph type.}{glyph.Rul.type.ggobiDataset}
\aliasA{glyph\_type}{glyph\_type.ggobiDataset}{glyph.Rul.type}
\keyword{dynamic}{glyph\_type.ggobiDataset}
\begin{Description}\relax
Get glyph type.
\end{Description}
\begin{Usage}
\begin{verbatim}glyph_type.ggobiDataset(x)\end{verbatim}
\end{Usage}
\begin{Arguments}
\begin{ldescription}
\item[\code{x}] ggobiDataset
\end{ldescription}
\end{Arguments}
\begin{Details}\relax
\end{Details}
\begin{Author}\relax
Hadley Wickham <h.wickham@gmail.com>
\end{Author}
\begin{SeeAlso}\relax
\code{\LinkA{glyph\_type<\Rdash}{glyph.Rul.type<.Rdash.}}
\end{SeeAlso}
\begin{Examples}
\begin{ExampleCode}\end{ExampleCode}
\end{Examples}

\HeaderA{"ids<-.ggobiDataset"}{Set row ids}{"ids<.Rdash..ggobiDataset"}
\aliasA{ids<\Rdash}{"ids<-.ggobiDataset"}{ids<.Rdash.}
\methaliasA{ids<\Rdash.ggobiDataset}{"ids<-.ggobiDataset"}{ids<.Rdash..ggobiDataset}
\keyword{manip}{"ids<-.ggobiDataset"}
\begin{Description}\relax
Set row ids from a ggobiDataset
\end{Description}
\begin{Usage}
\begin{verbatim}"ids<-.ggobiDataset"(x, value)\end{verbatim}
\end{Usage}
\begin{Arguments}
\begin{ldescription}
\item[\code{x}] ggobiDataset
\item[\code{value}] new values
\end{ldescription}
\end{Arguments}
\begin{Details}\relax
\end{Details}
\begin{Author}\relax
Hadley Wickham <h.wickham@gmail.com>
\end{Author}
\begin{SeeAlso}\relax
\code{\LinkA{ids}{ids}}
\end{SeeAlso}
\begin{Examples}
\begin{ExampleCode}\end{ExampleCode}
\end{Examples}

\HeaderA{ids.ggobiDataset}{Row ids}{ids.ggobiDataset}
\aliasA{ids}{ids.ggobiDataset}{ids}
\keyword{manip}{ids.ggobiDataset}
\begin{Description}\relax
Retrive row ids from a ggobiDataset
\end{Description}
\begin{Usage}
\begin{verbatim}ids.ggobiDataset(x)\end{verbatim}
\end{Usage}
\begin{Arguments}
\begin{ldescription}
\item[\code{x}] ggobiDataset
\end{ldescription}
\end{Arguments}
\begin{Details}\relax
\end{Details}
\begin{Author}\relax
Hadley Wickham <h.wickham@gmail.com>
\end{Author}
\begin{SeeAlso}\relax
\code{\LinkA{ids<\Rdash}{ids<.Rdash.}}
\end{SeeAlso}
\begin{Examples}
\begin{ExampleCode}\end{ExampleCode}
\end{Examples}

\HeaderA{mapGlyphType}{Map glyph type.}{mapGlyphType}
\keyword{dynamic}{mapGlyphType}
\keyword{internal}{mapGlyphType}
\begin{Description}\relax
Map glyph character code to number
\end{Description}
\begin{Usage}
\begin{verbatim}mapGlyphType(types)\end{verbatim}
\end{Usage}
\begin{Arguments}
\begin{ldescription}
\item[\code{types}] vector of glyph character codes
\end{ldescription}
\end{Arguments}
\begin{Details}\relax
\end{Details}
\begin{Author}\relax
Hadley Wickham <h.wickham@gmail.com>
\end{Author}
\begin{Examples}
\begin{ExampleCode}\end{ExampleCode}
\end{Examples}

\HeaderA{"names<-.ggobiDataset"}{Set column names}{"names<.Rdash..ggobiDataset"}
\aliasA{names<\Rdash.ggobiDataset}{"names<-.ggobiDataset"}{names<.Rdash..ggobiDataset}
\keyword{attribute}{"names<-.ggobiDataset"}
\keyword{internal}{"names<-.ggobiDataset"}
\begin{Description}\relax
Set column names for a ggobiDataset
\end{Description}
\begin{Usage}
\begin{verbatim}"names<-.ggobiDataset"(x, value)\end{verbatim}
\end{Usage}
\begin{Arguments}
\begin{ldescription}
\item[\code{x}] ggobiDataset
\item[\code{value}] new names
\end{ldescription}
\end{Arguments}
\begin{Details}\relax
\end{Details}
\begin{Author}\relax
Hadley Wickham <h.wickham@gmail.com>
\end{Author}
\begin{Examples}
\begin{ExampleCode}\end{ExampleCode}
\end{Examples}

\HeaderA{names.ggobi}{GGobi names}{names.ggobi}
\keyword{dynamic}{names.ggobi}
\begin{Description}\relax
Get dataset names
\end{Description}
\begin{Usage}
\begin{verbatim}names.ggobi(x)\end{verbatim}
\end{Usage}
\begin{Arguments}
\begin{ldescription}
\item[\code{x}] 
\end{ldescription}
\end{Arguments}
\begin{Details}\relax
\end{Details}
\begin{Author}\relax
Hadley Wickham <h.wickham@gmail.com>
\end{Author}
\begin{Examples}
\begin{ExampleCode}g <- ggobi(mtcars)
names(g)\end{ExampleCode}
\end{Examples}

\HeaderA{names.ggobiDataset}{ggobiDataset column names}{names.ggobiDataset}
\keyword{attribute}{names.ggobiDataset}
\keyword{internal}{names.ggobiDataset}
\begin{Description}\relax
Get column names for a ggobiDataset
\end{Description}
\begin{Usage}
\begin{verbatim}names.ggobiDataset(x, ...)\end{verbatim}
\end{Usage}
\begin{Arguments}
\begin{ldescription}
\item[\code{x}] dataset
\item[\code{...}] 
\end{ldescription}
\end{Arguments}
\begin{Details}\relax
\end{Details}
\begin{Author}\relax
Hadley Wickham <h.wickham@gmail.com>
\end{Author}
\begin{Examples}
\begin{ExampleCode}\end{ExampleCode}
\end{Examples}

\HeaderA{ncol.ggobiDataset}{ggobiDataset columns}{ncol.ggobiDataset}
\keyword{attribute}{ncol.ggobiDataset}
\keyword{internal}{ncol.ggobiDataset}
\begin{Description}\relax
Retrieve the number of columns in an ggobi dataset
\end{Description}
\begin{Usage}
\begin{verbatim}ncol.ggobiDataset(d)\end{verbatim}
\end{Usage}
\begin{Arguments}
\begin{ldescription}
\item[\code{d}] dataset
\end{ldescription}
\end{Arguments}
\begin{Details}\relax
\end{Details}
\begin{Author}\relax
Hadley Wickham <h.wickham@gmail.com>
\end{Author}
\begin{Examples}
\begin{ExampleCode}\end{ExampleCode}
\end{Examples}

\HeaderA{nrow.ggobiDataset}{ggobiDataset rows}{nrow.ggobiDataset}
\keyword{attribute}{nrow.ggobiDataset}
\keyword{internal}{nrow.ggobiDataset}
\begin{Description}\relax
Retrieve the number of row in an ggobi dataset
\end{Description}
\begin{Usage}
\begin{verbatim}nrow.ggobiDataset(d)\end{verbatim}
\end{Usage}
\begin{Arguments}
\begin{ldescription}
\item[\code{d}] dataset
\end{ldescription}
\end{Arguments}
\begin{Details}\relax
\end{Details}
\begin{Author}\relax
Hadley Wickham <h.wickham@gmail.com>
\end{Author}
\begin{Examples}
\begin{ExampleCode}\end{ExampleCode}
\end{Examples}

\HeaderA{print.ggobi}{Print ggobi}{print.ggobi}
\keyword{dynamic}{print.ggobi}
\keyword{internal}{print.ggobi}
\begin{Description}\relax
Prints summary of ggobi object by instance
\end{Description}
\begin{Usage}
\begin{verbatim}print.ggobi(x, ...)\end{verbatim}
\end{Usage}
\begin{Arguments}
\begin{ldescription}
\item[\code{x}] ggobi object
\item[\code{...}] 
\end{ldescription}
\end{Arguments}
\begin{Details}\relax
\end{Details}
\begin{Author}\relax
Hadley Wickham <h.wickham@gmail.com>
\end{Author}
\begin{SeeAlso}\relax
\code{\LinkA{summary.ggobi}{summary.ggobi}}
\end{SeeAlso}
\begin{Examples}
\begin{ExampleCode}\end{ExampleCode}
\end{Examples}

\HeaderA{print.ggobiDataset}{Print ggobiDataset}{print.ggobiDataset}
\keyword{attribute}{print.ggobiDataset}
\keyword{internal}{print.ggobiDataset}
\begin{Description}\relax
Print ggobiDataset
\end{Description}
\begin{Usage}
\begin{verbatim}print.ggobiDataset(x, ...)\end{verbatim}
\end{Usage}
\begin{Arguments}
\begin{ldescription}
\item[\code{x}] GGobi dataset to retrieve
\item[\code{...}] 
\end{ldescription}
\end{Arguments}
\begin{Details}\relax
By default printing a ggobiDataset acts like
printing an R data.frame - ie. show all the data
\end{Details}
\begin{Author}\relax
Hadley Wickham <h.wickham@gmail.com>
\end{Author}
\begin{Examples}
\begin{ExampleCode}\end{ExampleCode}
\end{Examples}

\HeaderA{"rownames<-.ggobiDataset"}{Set row names}{"rownames<.Rdash..ggobiDataset"}
\aliasA{rownames<\Rdash.ggobiDataset}{"rownames<-.ggobiDataset"}{rownames<.Rdash..ggobiDataset}
\keyword{attribute}{"rownames<-.ggobiDataset"}
\keyword{internal}{"rownames<-.ggobiDataset"}
\begin{Description}\relax
Set row names for a ggobiDataset
\end{Description}
\begin{Usage}
\begin{verbatim}"rownames<-.ggobiDataset"(x, value)\end{verbatim}
\end{Usage}
\begin{Arguments}
\begin{ldescription}
\item[\code{x}] ggobiDataset
\item[\code{value}] new names
\end{ldescription}
\end{Arguments}
\begin{Details}\relax
\end{Details}
\begin{Author}\relax
Hadley Wickham <h.wickham@gmail.com>
\end{Author}
\begin{Examples}
\begin{ExampleCode}\end{ExampleCode}
\end{Examples}

\HeaderA{rownames.ggobiDataset}{Get row names}{rownames.ggobiDataset}
\keyword{attribute}{rownames.ggobiDataset}
\keyword{internal}{rownames.ggobiDataset}
\begin{Description}\relax
Get row names for a ggobiDataget
\end{Description}
\begin{Usage}
\begin{verbatim}rownames.ggobiDataset(x)\end{verbatim}
\end{Usage}
\begin{Arguments}
\begin{ldescription}
\item[\code{x}] ggobiDataget
\item[\code{}] 
\end{ldescription}
{new names}
\end{Arguments}
\begin{Details}\relax
\end{Details}
\begin{Author}\relax
Hadley Wickham <h.wickham@gmail.com>
\end{Author}
\begin{Examples}
\begin{ExampleCode}\end{ExampleCode}
\end{Examples}

\HeaderA{selected.ggobiDataset}{Get selection status}{selected.ggobiDataset}
\aliasA{selected}{selected.ggobiDataset}{selected}
\keyword{dynamic}{selected.ggobiDataset}
\begin{Description}\relax
Returns logical vector indicating if each point is under the brush
\end{Description}
\begin{Usage}
\begin{verbatim}selected.ggobiDataset(x)\end{verbatim}
\end{Usage}
\begin{Arguments}
\begin{ldescription}
\item[\code{x}] ggobiDataset
\item[\code{}] 
\end{ldescription}
{logical vector}
\end{Arguments}
\begin{Details}\relax
\end{Details}
\begin{Author}\relax
Hadley Wickham <h.wickham@gmail.com>
\end{Author}
\begin{Examples}
\begin{ExampleCode}\end{ExampleCode}
\end{Examples}

\HeaderA{"[<-.ggobi"}{[<-.ggobi}{"[<.Rdash..ggobi"}
\aliasA{\$<\Rdash.ggobi}{"[<-.ggobi"}{.Rdol.<.Rdash..ggobi}
\aliasA{[<\Rdash.ggobi}{"[<-.ggobi"}{[<.Rdash..ggobi}
\keyword{manip}{"[<-.ggobi"}
\begin{Description}\relax
Add data to ggobi instance.
\end{Description}
\begin{Usage}
\begin{verbatim}"[<-.ggobi"(x, i, value)\end{verbatim}
\end{Usage}
\begin{Arguments}
\begin{ldescription}
\item[\code{x}] ggobi instance
\item[\code{i}] name of data frame
\item[\code{value}] data.frame, or string to path of file to load
\end{ldescription}
\end{Arguments}
\begin{Details}\relax
This function allows you to add (and eventually) replace
ggobi to a ggobi instance.
\end{Details}
\begin{Author}\relax
Hadley Wickham <h.wickham@gmail.com>
\end{Author}
\begin{Examples}
\begin{ExampleCode}g <- ggobi()
g["a"] <- mtcars
g$b <- mtcars\end{ExampleCode}
\end{Examples}

\HeaderA{"[<-.ggobiDataset"}{Assignments for ggobi datasets}{"[<.Rdash..ggobiDataset"}
\aliasA{[<\Rdash.ggobiDataset}{"[<-.ggobiDataset"}{[<.Rdash..ggobiDataset}
\keyword{manip}{"[<-.ggobiDataset"}
\keyword{internal}{"[<-.ggobiDataset"}
\begin{Description}\relax
\end{Description}
\begin{Usage}
\begin{verbatim}"[<-.ggobiDataset"(x, i, j, value)\end{verbatim}
\end{Usage}
\begin{Arguments}
\begin{ldescription}
\item[\code{x}] row indices
\item[\code{i}] column indices
\item[\code{j}] new values
\item[\code{value}] 
\end{ldescription}
\end{Arguments}
\begin{Details}\relax
This functions allow one to treat a ggobi dataset as if it were a local
data.frame.  One can extract and assign elements within the dataset.

This method works by retrieving the entire dataset into
R, subsetting that copy, and then returning any changes.

@argument ggobi dataset
@arguments row indices
@arguments column indices
@arguments new values
@keyword manip
@keyword internal
\end{Details}
\begin{Author}\relax
Hadley Wickham <h.wickham@gmail.com>
\end{Author}
\begin{Examples}
\begin{ExampleCode}g <- ggobi(mtcars)
x <- g["mtcars"]
x[1:5, 1:5]
x[1:5, 1] <- 1:5
x[1:5, 1:5]\end{ExampleCode}
\end{Examples}

\HeaderA{"shadowed<-.ggobiDataset"}{Set shadowed status}{"shadowed<.Rdash..ggobiDataset"}
\aliasA{shadowed<\Rdash}{"shadowed<-.ggobiDataset"}{shadowed<.Rdash.}
\methaliasA{shadowed<\Rdash.ggobiDataset}{"shadowed<-.ggobiDataset"}{shadowed<.Rdash..ggobiDataset}
\keyword{dynamic}{"shadowed<-.ggobiDataset"}
\begin{Description}\relax
Set the exclusion status of points.
\end{Description}
\begin{Usage}
\begin{verbatim}"shadowed<-.ggobiDataset"(x, value)\end{verbatim}
\end{Usage}
\begin{Arguments}
\begin{ldescription}
\item[\code{x}] ggobiDataset
\item[\code{value}] logical vector
\end{ldescription}
\end{Arguments}
\begin{Details}\relax
If a point is shadowed it is drawn in a dark gray colour.
\end{Details}
\begin{Author}\relax
Hadley Wickham <h.wickham@gmail.com>
\end{Author}
\begin{SeeAlso}\relax
\code{\LinkA{shadowed}{shadowed}}
\end{SeeAlso}
\begin{Examples}
\begin{ExampleCode}\end{ExampleCode}
\end{Examples}

\HeaderA{shadowed.ggobiDataset}{Get shadowed status}{shadowed.ggobiDataset}
\aliasA{shadowed}{shadowed.ggobiDataset}{shadowed}
\keyword{dynamic}{shadowed.ggobiDataset}
\begin{Description}\relax
Get the exclusion status of points.
\end{Description}
\begin{Usage}
\begin{verbatim}shadowed.ggobiDataset(x)\end{verbatim}
\end{Usage}
\begin{Arguments}
\begin{ldescription}
\item[\code{x}] ggobiDataget
\end{ldescription}
\end{Arguments}
\begin{Details}\relax
If a point is shadowed it is drawn in a dark gray colour.
\end{Details}
\begin{Author}\relax
Hadley Wickham <h.wickham@gmail.com>
\end{Author}
\begin{SeeAlso}\relax
\code{\LinkA{shadowed<\Rdash}{shadowed<.Rdash.}}
\end{SeeAlso}
\begin{Examples}
\begin{ExampleCode}\end{ExampleCode}
\end{Examples}

\HeaderA{summary.ggobi}{GGobi summary}{summary.ggobi}
\keyword{dynamic}{summary.ggobi}
\begin{Description}\relax
Get a description of the global state of the GGobi session.
\end{Description}
\begin{Usage}
\begin{verbatim}summary.ggobi(object, ...)\end{verbatim}
\end{Usage}
\begin{Arguments}
\begin{ldescription}
\item[\code{object}] ggobi object
\item[\code{...}] 
\end{ldescription}
\end{Arguments}
\begin{Details}\relax
\end{Details}
\begin{Author}\relax
Hadley Wickham <h.wickham@gmail.com>
\end{Author}
\begin{Examples}
\begin{ExampleCode}g <- ggobi(mtcars)
summary(g)\end{ExampleCode}
\end{Examples}

\HeaderA{summary.ggobiDataset}{Summarise ggobiDataset.}{summary.ggobiDataset}
\keyword{attribute}{summary.ggobiDataset}
\begin{Description}\relax
Summarise a ggobiDataset with dimensions, mode and variable names.
\end{Description}
\begin{Usage}
\begin{verbatim}summary.ggobiDataset(object, ...)\end{verbatim}
\end{Usage}
\begin{Arguments}
\begin{ldescription}
\item[\code{object}] ggobiDataset
\item[\code{...}] 
\end{ldescription}
\end{Arguments}
\begin{Details}\relax
\end{Details}
\begin{Author}\relax
Hadley Wickham <h.wickham@gmail.com>
\end{Author}
\begin{Examples}
\begin{ExampleCode}\end{ExampleCode}
\end{Examples}

\HeaderA{valid\_ggobi}{Determines whether a reference to an internal ggobi object is valid}{valid.Rul.ggobi}
\keyword{dynamic}{valid\_ggobi}
\keyword{internal}{valid\_ggobi}
\begin{Description}\relax
\end{Description}
\begin{Usage}
\begin{verbatim}valid_ggobi(.gobi)\end{verbatim}
\end{Usage}
\begin{Arguments}
\begin{ldescription}
\item[\code{.gobi}] an object of class \code{ggobi} which refers to an internal ggobi instance.
\end{ldescription}
\end{Arguments}
\begin{Details}\relax
One can create multiple, independent ggobi instances within a single
R session and one can also remove them either programmatically or
via the GUI.  To be able to refer to these objects which are
actually C-level internal objects, one has a reference or handle
from an S object. Since the C level object can be destroyed while the S
object still refers to them, this function allows one to check whether the
internal object to which R refers is still in existence.

@arguments an object of class \code{ggobi} which refers to an internal ggobi instance.
@value \code{TRUE} if real object still exist, \code{FALSE} otherwise
@keyword dynamic
\end{Details}
\begin{Value}
\code{TRUE} if real object still exist, \code{FALSE} otherwise
\end{Value}
\begin{Author}\relax
Hadley Wickham <h.wickham@gmail.com>
\end{Author}
\begin{Examples}
\begin{ExampleCode}g <- ggobi(mtcars)
valid_ggobi(g)
close(g)
valid_ggobi(g) \end{ExampleCode}
\end{Examples}

\HeaderA{variable\_index}{Variable index}{variable.Rul.index}
\keyword{attribute}{variable\_index}
\keyword{internal}{variable\_index}
\begin{Description}\relax
Return indices corresponding to variable names
\end{Description}
\begin{Usage}
\begin{verbatim}variable_index(x, names)\end{verbatim}
\end{Usage}
\begin{Arguments}
\begin{ldescription}
\item[\code{x}] ggobiDataset
\item[\code{names}] variable names
\end{ldescription}
\end{Arguments}
\begin{Details}\relax
\end{Details}
\begin{Author}\relax
Hadley Wickham <h.wickham@gmail.com>
\end{Author}
\begin{Examples}
\begin{ExampleCode}\end{ExampleCode}
\end{Examples}

\end{document}
